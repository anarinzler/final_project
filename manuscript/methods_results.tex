\documentclass[man]{apa6}

\usepackage{amssymb,amsmath}
\usepackage{ifxetex,ifluatex}
\usepackage{fixltx2e} % provides \textsubscript
\ifnum 0\ifxetex 1\fi\ifluatex 1\fi=0 % if pdftex
  \usepackage[T1]{fontenc}
  \usepackage[utf8]{inputenc}
\else % if luatex or xelatex
  \ifxetex
    \usepackage{mathspec}
    \usepackage{xltxtra,xunicode}
  \else
    \usepackage{fontspec}
  \fi
  \defaultfontfeatures{Mapping=tex-text,Scale=MatchLowercase}
  \newcommand{\euro}{€}
\fi
% use upquote if available, for straight quotes in verbatim environments
\IfFileExists{upquote.sty}{\usepackage{upquote}}{}
% use microtype if available
\IfFileExists{microtype.sty}{\usepackage{microtype}}{}

% Table formatting
\usepackage{longtable, booktabs}
\usepackage{lscape}
% \usepackage[counterclockwise]{rotating}   % Landscape page setup for large tables
\usepackage{multirow}		% Table styling
\usepackage{tabularx}		% Control Column width
\usepackage[flushleft]{threeparttable}	% Allows for three part tables with a specified notes section
\usepackage{threeparttablex}            % Lets threeparttable work with longtable

% Create new environments so endfloat can handle them
% \newenvironment{ltable}
%   {\begin{landscape}\begin{center}\begin{threeparttable}}
%   {\end{threeparttable}\end{center}\end{landscape}}

\newenvironment{lltable}
  {\begin{landscape}\begin{center}\begin{ThreePartTable}}
  {\end{ThreePartTable}\end{center}\end{landscape}}

  \usepackage{ifthen} % Only add declarations when endfloat package is loaded
  \ifthenelse{\equal{\string man}{\string man}}{%
   \DeclareDelayedFloatFlavor{ThreePartTable}{table} % Make endfloat play with longtable
   % \DeclareDelayedFloatFlavor{ltable}{table} % Make endfloat play with lscape
   \DeclareDelayedFloatFlavor{lltable}{table} % Make endfloat play with lscape & longtable
  }{}%



% The following enables adjusting longtable caption width to table width
% Solution found at http://golatex.de/longtable-mit-caption-so-breit-wie-die-tabelle-t15767.html
\makeatletter
\newcommand\LastLTentrywidth{1em}
\newlength\longtablewidth
\setlength{\longtablewidth}{1in}
\newcommand\getlongtablewidth{%
 \begingroup
  \ifcsname LT@\roman{LT@tables}\endcsname
  \global\longtablewidth=0pt
  \renewcommand\LT@entry[2]{\global\advance\longtablewidth by ##2\relax\gdef\LastLTentrywidth{##2}}%
  \@nameuse{LT@\roman{LT@tables}}%
  \fi
\endgroup}


\ifxetex
  \usepackage[setpagesize=false, % page size defined by xetex
              unicode=false, % unicode breaks when used with xetex
              xetex]{hyperref}
\else
  \usepackage[unicode=true]{hyperref}
\fi
\hypersetup{breaklinks=true,
            pdfauthor={},
            pdftitle={Effects of perceptual training on ability to produce L2 English plosives},
            colorlinks=true,
            citecolor=blue,
            urlcolor=blue,
            linkcolor=black,
            pdfborder={0 0 0}}
\urlstyle{same}  % don't use monospace font for urls

\setlength{\parindent}{0pt}
%\setlength{\parskip}{0pt plus 0pt minus 0pt}

\setlength{\emergencystretch}{3em}  % prevent overfull lines


% Manuscript styling
\captionsetup{font=singlespacing,justification=justified}
\usepackage{csquotes}
\usepackage{upgreek}

 % Line numbering
  \usepackage{lineno}
  \linenumbers


\usepackage{tikz} % Variable definition to generate author note

% fix for \tightlist problem in pandoc 1.14
\providecommand{\tightlist}{%
  \setlength{\itemsep}{0pt}\setlength{\parskip}{0pt}}

% Essential manuscript parts
  \title{Effects of perceptual training on ability to produce L2 English plosives}

  \shorttitle{Production of L2 plosives}


  \author{Ana Noelle Bennett}

  % \def\affdep{{""}}%
  % \def\affcity{{""}}%

  \affiliation{
    \vspace{0.5cm}
          \textsuperscript{} Rutgers, the State University of New Jersey  }

  \authornote{
    Cognitive Psychology\\
    Center for Cognitive Science
    
    Correspondence concerning this article should be addressed to Ana Noelle
    Bennett, 1498 State Route 28 West Hurely, NY 12491. E-mail:
    \href{mailto:anb136@psych.rutgers.edu}{\nolinkurl{anb136@psych.rutgers.edu}}
  }


  \abstract{Enter abstract here. Each new line herein must be indented, like this
line.}
  \keywords{keywords \\

    \indent Word count: X
  }





\usepackage{amsthm}
\newtheorem{theorem}{Theorem}[section]
\newtheorem{lemma}{Lemma}[section]
\theoremstyle{definition}
\newtheorem{definition}{Definition}[section]
\newtheorem{corollary}{Corollary}[section]
\newtheorem{proposition}{Proposition}[section]
\theoremstyle{definition}
\newtheorem{example}{Example}[section]
\theoremstyle{definition}
\newtheorem{exercise}{Exercise}[section]
\theoremstyle{remark}
\newtheorem*{remark}{Remark}
\newtheorem*{solution}{Solution}
\begin{document}

\maketitle

\setcounter{secnumdepth}{0}



\section{Methods}\label{methods}

In this study, we conducted acoustic analyses on Cantonese speakers'
productions of English phonological minimal word pairs with voiced (i.e.
/b d g/) and voiceless (i.e. /p t k/) plosives in coda position.
Specifically, we utilized PRAAT to measure the duration of the vowel.
This was motivated by evidence that vowel duration is an acoustic cue
that indicates in English if the following plosive is voiced or
voiceless -- with the duration of the vowel preceding a voiced stop
being longer than the duration of a vowel preceding a voiceless stop
(Charles-Luce, 1985; House and Fairbanks, 1953; Peterson and Lehiste,
1960; House, 1961; Umeda, 1975; Klatt, 1976). The production of the word
\enquote{got} was excluded from this analysis as it was the only word
that did not have a minimal pair.Please note that the production data
analyzed in this study was collected and generously provided by
Dr.~Terry Kit-fong Au, from the University of Hong Kong.

\subsection{Participants}\label{participants}

There were a total of 36 University students from the University of Hong
Kong. 18 of the participants were in the training group (33\% men), and
18 of the participants were in a wait-list control group (28\% men).

\subsection{Material}\label{material}

The following analyses are based on productions of phonological minimal
word pairs with voiced and voiceless plosives in coda position. The
vowel duration from the following voiced words were analyzed:/b\ae d,
bæg, k\ae b, k\textturnv b , d\textopeno g , f\ae d, fid, p\textsci g,
t\ae b/. The following voiceless words were analyzed: / b\ae t, b\ae k,
k\ae p, k\textturnv p, d\textscripta k, f\ae t, fit, p\textsci k,
t\ae b/. Only \enquote{post-training} productions were analyzed. For the
wait-list control particpants, this was the second time that they
produced these words (e.g.~they did receive training in between the
first and second times that they produced these words). However for
trained participants, these productions represent the second time that
they produced these words after they received intensive training.

\subsection{Procedure}\label{procedure}

Participants in Terry Au's (ms) study participated in a 4 - 6 week
training program compromised of comprehending and producing English
phonological minmial word pairs. Not all of the words that were used in
training were used in production. See \emph{Appendix A} for full list of
words, as well as which words were used in training, and which were not.
The productions were then sent to our lab for acoustic analyses.

The software PRAAT was used to conduct acoustic analyses. Textgrids were
created from the sound file in order to mark the beginning and end of
the vowel boundary. Utilizing Sennheiser HD 555 headphones, the
beginning of the vowel was marked with the \emph{wav} method and the end
of the vowel was marked with the \emph{F2} method. All boundaries were
marked at the zero-crossing line. Measurements at present, were only
taken by one researcher. Thus, future cross-validation through
concordance rates is required. PRAAT scripting was then used to export
vowel duration measurements.

\subsection{Data analysis}\label{data-analysis}

We used R (Version 3.4.3; R Core Team, 2017) and the R-packages
\emph{bindrcpp} (Version 0.2.2; Müller, 2018), \emph{broom} (Version
0.4.4; Robinson, 2018), \emph{doBy} (Version 4.6.1; Højsgaard \&
Halekoh, 2018), \emph{dplyr} (Version 0.7.4; Wickham, Francois, Henry,
\& Müller, 2017), \emph{forcats} (Version 0.3.0; Wickham, 2018a),
\emph{ggfortify} (Version 0.4.4; Tang, Horikoshi, \& Li, 2016),
\emph{ggplot2} (Version 2.2.1; Wickham, 2009), \emph{kableExtra}
(Version 0.8.0; Zhu, 2018), \emph{likelihood} (Version 1.7; Murphy,
2015), \emph{lme4} (Version 1.1.17; Bates, Mächler, Bolker, \& Walker,
2015), \emph{lmerTest} (Version 3.0.1; Kuznetsova, Brockhoff, \&
Christensen, 2017), \emph{Matrix} (Version 1.2.14; Bates \& Maechler,
2018), \emph{MuMIn} (Version 1.40.4; Bartoń, 2018), \emph{nlme} (Version
3.1.137; Pinheiro, Bates, DebRoy, Sarkar, \& R Core Team, 2018),
\emph{papaja} (Version 0.1.0.9709; Aust \& Barth, 2018), \emph{purrr}
(Version 0.2.4; Henry \& Wickham, 2017), \emph{readr} (Version 1.1.1;
Wickham, Hester, \& Francois, 2017), \emph{stringr} (Version 1.3.0;
Wickham, 2018b), \emph{tibble} (Version 1.4.2; Müller \& Wickham, 2018),
\emph{tidyr} (Version 0.8.0; Wickham \& Henry, 2018), \emph{tidyverse}
(Version 1.2.1; Wickham, 2017), and \emph{xaringan} (Version 0.6.4; Xie,
n.d.) for all our analyses. Data from the production task were analyzed
using a general linear mixed-effects model using the lme4 package
(1.1-10 in R 3.2.2). The criterion variable was \emph{vowel duration}
which was convereted to milliseconds and normalized for speaker. There
were two predictors which were fixed factors: (1) training
\emph{trained/untrained} and voicing \emph{voiced/unvoiced}. Both
factors were cateogrical and were sum coded. For the training variable,
\emph{trained} was assigned a 1, and \emph{untrained} was assigned a 0;
while \emph{voiced} was assigned a 1 and \emph{voiceless} was assigned a
0. Two new columns in the data frame were generated to represent the sum
variables of the training and the voicing variables. The variable
participant was treated as a random effect as each participant had
multiple productions (i.e.~each participant produced each of the 36
voiced and voiceless words). Visual inspection of the Q-Q plots and
plots of residuals against fitted values revealed that the assumptions
of normality and homoscedasticity were in tact.

\section{Discussion}\label{discussion}

\newpage

\section{References}\label{references}

\begingroup
\setlength{\parindent}{-0.5in} \setlength{\leftskip}{0.5in}

\hypertarget{refs}{}
\hypertarget{ref-R-papaja}{}
Aust, F., \& Barth, M. (2018). \emph{papaja: Create APA manuscripts with
R Markdown}. Retrieved from \url{https://github.com/crsh/papaja}

\hypertarget{ref-R-MuMIn}{}
Bartoń, K. (2018). \emph{MuMIn: Multi-model inference}. Retrieved from
\url{https://CRAN.R-project.org/package=MuMIn}

\hypertarget{ref-R-Matrix}{}
Bates, D., \& Maechler, M. (2018). \emph{Matrix: Sparse and dense matrix
classes and methods}. Retrieved from
\url{https://CRAN.R-project.org/package=Matrix}

\hypertarget{ref-R-lme4}{}
Bates, D., Mächler, M., Bolker, B., \& Walker, S. (2015). Fitting linear
mixed-effects models using lme4. \emph{Journal of Statistical Software},
\emph{67}(1), 1--48.
doi:\href{https://doi.org/10.18637/jss.v067.i01}{10.18637/jss.v067.i01}

\hypertarget{ref-R-purrr}{}
Henry, L., \& Wickham, H. (2017). \emph{Purrr: Functional programming
tools}. Retrieved from \url{https://CRAN.R-project.org/package=purrr}

\hypertarget{ref-R-doBy}{}
Højsgaard, S., \& Halekoh, U. (2018). \emph{DoBy: Groupwise statistics,
lsmeans, linear contrasts, utilities}. Retrieved from
\url{https://CRAN.R-project.org/package=doBy}

\hypertarget{ref-R-lmerTest}{}
Kuznetsova, A., Brockhoff, P. B., \& Christensen, R. H. B. (2017).
lmerTest package: Tests in linear mixed effects models. \emph{Journal of
Statistical Software}, \emph{82}(13), 1--26.
doi:\href{https://doi.org/10.18637/jss.v082.i13}{10.18637/jss.v082.i13}

\hypertarget{ref-R-likelihood}{}
Murphy, L. (2015). \emph{Likelihood: Methods for maximum likelihood
estimation}. Retrieved from
\url{https://CRAN.R-project.org/package=likelihood}

\hypertarget{ref-R-bindrcpp}{}
Müller, K. (2018). \emph{Bindrcpp: An 'rcpp' interface to active
bindings}. Retrieved from
\url{https://CRAN.R-project.org/package=bindrcpp}

\hypertarget{ref-R-tibble}{}
Müller, K., \& Wickham, H. (2018). \emph{Tibble: Simple data frames}.
Retrieved from \url{https://CRAN.R-project.org/package=tibble}

\hypertarget{ref-R-nlme}{}
Pinheiro, J., Bates, D., DebRoy, S., Sarkar, D., \& R Core Team. (2018).
\emph{nlme: Linear and nonlinear mixed effects models}. Retrieved from
\url{https://CRAN.R-project.org/package=nlme}

\hypertarget{ref-R-base}{}
R Core Team. (2017). \emph{R: A language and environment for statistical
computing}. Vienna, Austria: R Foundation for Statistical Computing.
Retrieved from \url{https://www.R-project.org/}

\hypertarget{ref-R-broom}{}
Robinson, D. (2018). \emph{Broom: Convert statistical analysis objects
into tidy data frames}. Retrieved from
\url{https://CRAN.R-project.org/package=broom}

\hypertarget{ref-R-ggfortify}{}
Tang, Y., Horikoshi, M., \& Li, W. (2016). Ggfortify: Unified interface
to visualize statistical result of popular r packages. \emph{The R
Journal}, \emph{8}(2). Retrieved from
\url{https://journal.r-project.org/}

\hypertarget{ref-R-ggplot2}{}
Wickham, H. (2009). \emph{Ggplot2: Elegant graphics for data analysis}.
Springer-Verlag New York. Retrieved from \url{http://ggplot2.org}

\hypertarget{ref-R-tidyverse}{}
Wickham, H. (2017). \emph{Tidyverse: Easily install and load the
'tidyverse'}. Retrieved from
\url{https://CRAN.R-project.org/package=tidyverse}

\hypertarget{ref-R-forcats}{}
Wickham, H. (2018a). \emph{Forcats: Tools for working with categorical
variables (factors)}. Retrieved from
\url{https://CRAN.R-project.org/package=forcats}

\hypertarget{ref-R-stringr}{}
Wickham, H. (2018b). \emph{Stringr: Simple, consistent wrappers for
common string operations}. Retrieved from
\url{https://CRAN.R-project.org/package=stringr}

\hypertarget{ref-R-tidyr}{}
Wickham, H., \& Henry, L. (2018). \emph{Tidyr: Easily tidy data with
'spread()' and 'gather()' functions}. Retrieved from
\url{https://CRAN.R-project.org/package=tidyr}

\hypertarget{ref-R-dplyr}{}
Wickham, H., Francois, R., Henry, L., \& Müller, K. (2017). \emph{Dplyr:
A grammar of data manipulation}. Retrieved from
\url{https://CRAN.R-project.org/package=dplyr}

\hypertarget{ref-R-readr}{}
Wickham, H., Hester, J., \& Francois, R. (2017). \emph{Readr: Read
rectangular text data}. Retrieved from
\url{https://CRAN.R-project.org/package=readr}

\hypertarget{ref-R-xaringan}{}
Xie, Y. (n.d.). \emph{Xaringan: Presentation ninja}. Retrieved from
\url{https://github.com/yihui/xaringan}

\hypertarget{ref-R-kableExtra}{}
Zhu, H. (2018). \emph{KableExtra: Construct complex table with 'kable'
and pipe syntax}. Retrieved from
\url{https://CRAN.R-project.org/package=kableExtra}

\endgroup






\end{document}
